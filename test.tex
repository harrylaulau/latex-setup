\documentclass[12pt]{article}
\usepackage[utf8]{inputenc}

% Math packages
\usepackage{amsmath, amssymb, amsthm}

% Image packages
\usepackage{graphicx, float}
\graphicspath{{images/}}

% Customization ------

% GEOMETRY
% sets paper size, margins, etc

\usepackage[letterpaper, top=1.0in, bottom=1.0in, left=1.20in, right=1.20in, heightrounded]{geometry}

% Line height
\renewcommand{\baselinestretch}{1.15}

% Paragraph skip and paragraph indent
\setlength{\parindent}{0pt}
\setlength{\parskip}{0.8em}

% --------------------


\title{Latex Test}
\author{Author Name}
\date{June 2025}



\begin{document}

\maketitle

\section{Section}

\subsection{Subsection}
This is the main body of text. \\
\textbackslash{} \textbackslash{} means new line

A newline is needed to separate paragraphs.


\newpage

\section{Section}

\subsection{Subsection}



\LaTeX{}
Inline Equation: $E=mc^2$

\begin{equation}
    5+5=10
\end{equation}

\begin{equation}
    \begin{split}
        A & = \frac{\pi r^2}{2}   \\
        A & = \frac{1}{2} \pi r^2
    \end{split}
\end{equation}

\begin{figure}[H]
    \centering
    \includegraphics[width=2in]{latex.png}
    \caption{Caption of Image}\label{fig:my_label}
\end{figure}

\end{document}

